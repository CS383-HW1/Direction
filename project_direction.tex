\documentclass[12pt]{report}
\usepackage{geometry}
\usepackage{titling}

\geometry{margin=1in}

\newcommand{\subtitle}[1]{%
	\posttitle{%
		\par\end{center}
		\begin{center}\large#1\end{center}
		\vskip0.5em}%
}

\title{Project Design and Direction}
\subtitle{CS 383 - Homework 1 - Group 3}
\author{Mason Fabel, Ronald Rodriguez, Lance Wells, Morgan Holbart, Zachary Yama, NAMES} % TODO: Everyone add your name to this list
\date{\today}

\begin{document}

\maketitle

\chapter{Individual Brainstorming}

% TODO: Each person needs a section here.

\begin{section}{Mason Fabel}
Our project is to be ``a turn-based co-operative multiplayer
'dungeon-crawl' role-playing game'', combining the ideas of an adventurous
science fiction setting and the premise of a lowly office worker rising
through the ranks to become the master of his own destiny.

While contemplating these goals, I came up with a few central concepts and
observations I would like to see incorporated into the final project.

\begin{subsection}{Tone and Theme}
While there are a lot of directions the overall tone and feel of this projecan take, I believe the best direction would be a lighthearted combination of comedy, in the vein of Douglass Adam's Hitchhiker's Guide, and
action/adventure, such as in Star Wars IV.

Taking this approach instead of a realistic approach gives the team more
freedom as to the shape of the final project. As we are not constrained by
accurately simulating reality, we can instead focus on creating a game that
is fun to play and avoid complicating areas that would not benifit from
such attentions.

Keeping a light tone will allow us to avoid dark or overly thoughtful
material. While this things have a time and a place, the purpose
of this project is to learn how to organize and create large pieces of
software, not to create an artistic of philosophical statement. By avoiding
such material, the focus is kept on the engineering, where it should be,
instead of the content, which is a matter for a different class.
\end{subsection}

\begin{subsection}{Content Generation}
The main purpose of this class is to learn software engineering, or how
to organize and build large software projects. While we have chosen to do
this in the form of a game, the purpose of this class is not game design,
but rather building a large application. Thus, whenever possible this
project should be moved away from creating game content and towards writing
code and organizing development.

To this end, I believe that as much game content should be generated on the
fly by the application as possible. This obviously covers simple things,
such as level layout, but it is not a hug jump of the immagination to take
this same aproach to many, if not all, of the elements that will appear in
the final product. Such examples may include NPCs, quest lines, or
equipment, but wherever we wish to have more than a handful of types this
sort of system is theoretically possible.

In addition to taking the burden of content generation away from the group
and giving it to the computer, this will have the further advantage of
allowing the size of the project to be scaled up (or down) by creating
different content generation methods for different sections of the final
game.

Finally, generating content will allow the final project, which is going to
be created by a small team in a short period of time, to have a lot of
content relative to programmer effort. Some finite amount of effort will
allow the final product to have a much larger amount of content, as
opposed to the more linear relationship involved when humans create
content by hand.
\end{subsection}

\begin{subsection}{NPC Social Networking}
While the previous sections discuss some general trends or directions I
like the final project to take after, this section contains my main idea
that I would very much like to see make it into the final project.

Traditionally, most RPGs have used a system based upon levels and
experience points. While this works well in a lot of settings, it seems to
me that this transfers poorly over to the concept of the main character
being an office worker.

Thus, I would like to make progression in our final project not dependant
upone collect experience or gaining levels, but rather on a ``social
network'' of NPCs, for lack of a better term.

The basic idea is pretty simple. You have some ``relationship value''
with each NPC character. This value can be some positive or negative value,
with positive values meaning they like the player and a negative value
meaning they do not.

In addition to relating to the player, NPCs would relate to each other by
some percentage value between $-100\%$ and $100\%$. This would then serve
to modify the player's relationship with that NPC.

For example, the player has a $+50$ relationship with NPC A and a $-15$
relationship with NPC B. NPC A has a $10\%$ relationship with NPC B, and
NPC B has a $50\%$ relationship with NPC A. Thus, the player's final
relationship value with NPC A is $50+0.1(-15)=48.5$, and the player's
final relationship value with NPC B is $-15+0.5(50)=10$.

As the number of NPCs increases, a increasingly complex and interesting web
of characters can be created.

Now suppose that NPCs were tied to plot events or player progression.
Specifically, in order to advance further in the game, the player needs
to be promoted within the office or organization they work for, and this
can only be accomplished by gathering favor with their coworkers and
managers.

Furthermore, NPCs could be made to ignore the player until they reached
some relationship threshhold, thus causing the player to build relationship
with that NPS's ``friends'' in order to interact with them.

Finally, some method of gaining relationship with NPCs would need to be
implemented, whether a traditional ``kill 30 grues'' quest or some other,
more original, method.

The result of this would be a varied and dynamic system (especially if this
network of NPCs was also generated by the game instead of being hard coded)
where the player is required to gain social goodwill until they finally
come to the attention of the top entities, thus recieving the final
promotion to CEO, or the final boss fight, or whatever the end game turns
out to be.
\end{subsection}

\end{section}

\begin{section}{Ronnie Rodriguez}
\begin{subsection}{Tone and Theme}
As my collegue has stated, a solid mix of humor and sci-fi action would
be an excellent direction to take the game in, especially since we have been asked to 
keep the content "G and PG" rated. 

This begs the question, though, as to what brand of humor we should
be trying to use. In many comedic workplace-set films and tv shows like 
The Office, Office Space, Parks and Recreation, IT Crowd, etc. a palpable type of 
sarcastic, droll, deadpan humor is used. It's almost as if the characters in 
these shows have just had all the life sucked out of them by their boring, 
unfulfilling work, and almost everything that exits their mouths is 
a bitter, sarcastic quip at something or someone. I think this type of humor 
could translate well into a corporate-office set game. I think it is entirely
possible to do this, while keeping it PG rated. 
\end{subsection}

\begin{subsection}{Health Bar}
In many hack-n-slash RPG games, increasing in level gives you a bonus to 
your health bar. I think it would be cool to implement such a feature in our game, 
but with a comedic twist. 

Assuming we adopt the deadpan style of humor popular in workplace-set comedies populated
by characters who, for all intents and purposes, seem to have all but lost their will 
to go on, I think that instead of a "health" bar, it would be funny to have "will to go on" bar. 

For instance, instead of saying something like "Health has been increased by 5" after going 
up a level, we could have something like "Your will to go on has slightly increased. So you got that going for you,
which is nice..." after our character receives a promotion within the company.
\end{subsection}
\end{section}

\begin{section}{Lance Wells}
The prompt in question asks that we merge two concepts:
\begin{itemize}
\item \emph{A Star-Warsish Sci-Fi in which you overthrow the evil empire}
\item \emph{A lowly office worker fending off paperwork and bureaucracy to challenge the CEO}
\end{itemize}

Blending the two ideas brings about a few questions:
Evil empire and a corporation to challenge? Why not work for the "bad 
guys"? Paperwork to fend off and a Star-Warsish Sci-Fi setting? Why not fend off 
sentient paperwork?

I also have a general storyline-concept that I would like to propose:\newline
\begin{center}
\emph{It's the year 272820 and the remnants of the human race have been enslaved by 
the Conquesting and Expanding Organization, otherwise known as the CEO.}
\end{center}

The only clear goal of the CEO is to mitigate the human spirit, and assimilate the entirety 
of humanity into its dull corporation through the most horrifying weapon imaginable - paperwork.

Maintaining a "Will to Live" (as per Ron's idea) is essential to surviving the work-ridden 
future. Only by fighting the MAN (Malevolent Autonomous Navigator), a super-intelligence deigned 
to direct the future of the Earth towards a grim and monotonous future can the player(s) rescue 
the Earth and avoid becoming soulless engines to a galactic war machine.
\end{section}

\begin{section}{Morgan Holbart Game Idea}
\begin{subsection}{Setting}
As a cooperative or, at least, multiplayer game, the story and overlying setting of the game is not very important,
I think the best blend of the predetermined sci-fi and bureaucracy themes would simply place the player in a world
that combine the two with no overt explanation. e.g. the player is an office worker in an alien immigration building.
This gives a setting, a floor based "dungeon" (the building), and easily changeable enemies. The enemies can be
any assortment of the aliens that come through the place, which means they can look like or do anything we decide to
fit the game, instead of trying to stick to a certain lore.
\end{subsection}

\begin{subsection}{Game Design}
The game will pit a player and his office colleagues into the immigration office with the desire to exterminate all
of the aliens currently inhabiting the immigration office. You will work your way up starting from floor 1 to the 
top floor where the main boss will be. Each floor will have a multitude of mini bosses, enemies, loot rooms, and 
puzzles. Because of the procedural generation, each game will have an entirely different flow. The player and his
colleagues should have to work together to solve each floor, finding and killing the miniboss to unlock the next level,
and completing any puzzles to collect the superior loot to assist with the next floors.

Because the game is multiplayer, permadeath is a difficult feature to implement, no one wants to wait for the game to 
end, so instead every time a player dies, when the next floor is reached by his allies, he will be revived with a 
reduced maximum health, if someone dies 3 or some number of times, the next death will be permanent. Any ally can 
however revive a dead player and sacrifice his maximum health to do so.

The game will feature a variety of weapons, the basic weapons will be the human weapons, and will be the most common,
you can find alien weaponry off the corpses of your fallen foes which is the second tier of loot, and the third tier 
of loot will drop from puzzles and mini bosses. All loot will be tradeable between party members.

The progression system will be based purely on gear and floor number, there is no need for an XP system, the enemies
will scale with their floor, as will the quality of loot drops.
\end{subsection}

\begin{subsection}{Game Technology}
The game should have several mechanics that function to enhance the game and demonstrate our coding prowess. As
a dungeon crawler, procedural generation is a must. All levels of the game should be procedurally generated, as
should all items, possibly monster abilities/AI.

The game should adopt multiplayer systems for both LAN and internet based gameplay, it can implement lobby based
matchmaking, or direct IP connection, but should not require an authoritative server for controlling game logic. 
This will leave the game open to hacking and potential cheating, but without any leaderboards and the desire for
LAN support, this should be a non issue. Each client can most likely handle everything on its own, with one specific
host that will manage the communication between the two.

The game needs to have a sophisticated enough AI to support both singleplayer play (bot controlled allies) and enemies
with a alterable difficulty to keep the challenge for players of all skill level. The AI logic will have to be 
controlled by the host client.
\end{subsection}
\end{section}

\begin{section}{Game Direction – Zachary Yama}
\begin{subsection}{Story and Theme}

I very much like the story written by our first two game idea entires. Much moreso than
my own, but either way, here's my take on it.

The basic premise of our game must include both office worker themes and sci-fi themes. 
The morphed combinational theme that comes to mind is of a scientist who develops weapons 
of mass destruction for a company you, the office worker, have deemed unethical and evil. 
The corruption of the company you work for is so expansive that you dare not to leave out 
of fear. However, because you aren’t the only scientist, you find yourself caught up in the 
evil mad scientists’ scene when you really don’t want to be. The only co-worker you’ve been 
friends with and ever enjoyed working with has accidently unleashed an aggressive stimuli 
into the complex because of reasons unknown. His dying act of kindness was to give you the 
only vaccination. Your quest is to make it to ground level and out safely (Underground 
working up? Tall building working down?), find out what happened to your friend’s 
bio-aggression project, and wipe out all infected individuals. All along the way you will 
be engineering and developing new weapons, traps, and dispatching strategies to take out 
mobs, solve puzzles, and defeat bosses.

\end{subsection}
\begin{subsection}{Building Weapons}
An additional feature that I would like to propose is the way weapons are created and
generated within the game. A good equipment system that feels alive is something that
can really bring a lot to the table. Though random generation is an option, allowing
the player to build something of their very own is equally if not more powerful. It 
creates additional immersion and charm to the game, and invites the player to continue
somewhat soley for the sake of finding out what new cool weapon they can build by 
defeating the next enemy.

For weapons and items, having the ability to construct your own from parts you've found 
throughout your adventures is extremely interesting. I've seen this done before in the
game Loadout, which is a free third person shooter. It employs about 6 total parts for
any weapon, which there are about 10 varaiations of. Each part will modify the behavior
of the weapon in some way. For example, if you decided to use a scope you can zoom. If
you decide instead to use a lazer scope, now you've got a lazer to aim with. The system
would have to be altered to fit our game, but it wouldn't take that much effort to come
up with several weapon designs that are not only customizable, but fun to play with. 
Having this kind of game element would bring a lot of personal attachment with it, making
the player more immersed. The more fun the better, eh?

\end{subsection}
\end{section}


\chapter{Group Vision}

% TODO: Discuss Friday
% TODO: Appoint writers
% TODO: Actually have content here by Monday

\begin{section}{Setting}

\emph{Please note that all of this is tenative and largely an artifact of storytelling, 
if you have any suggestions or changes, \textbf{please make them}.}
\newline(law)
\newline

\noindent \emph{I have decided to write this section in terms of fictional prose in an 
effort to more accurately describe the scenario without sounding too laborious in the 
process. If you have another recommendation, feel free to comment as such.}
\newline(law)
\newline

Nestled between the dusty bosom of Mars and the neo-genetic warlords of Venus lies the 
glorious visage of Earth, a bustling, cybernetic utopia whose chief export - human labor 
- drives the many corporate houses calling the blue, green, and gray planet home.

One of the largest organizations on Earth, the Western Organization for Resurgent 
Knowledge (W.O.R.K.) is your newest employer. White-washed, colorless cubicles blanketed 
by flickering, fluorescent lights stretch for miles as your sleepless coworkers dawdle 
about their systems.

Fortunately, escape from this bleak business purgatory is but keystrokes away. As an 
Information Technology Associate (of the future), your primary purpose in this business 
is to hunt-down and remove any pesky viruses infecting the systems of your associates.
\newline

Neon-blue lines shift, pulsate, and shimmer as you enter the system of a sales associate 
from cubicle 28xA1x47, Martha. Gradually coming into focus, matrices of virtual wire 
outline your new surroundings. Walls and corridors begin to form and shift into a gradually 
expanding labyrinth. Just as you catch your breath, a dimly glowing red object comes into 
view, twitching menacingly.

``A virus,'' you mutter to yourself as your right hand gradually shifts downwards towards 
the weapon holstered on your e-belt. A lime-green, glittering pistol reaches your 
fingertips. Pieced together meticulously from parts won from previous conquests, its forking 
barrel snaps into action. As you pull the trigger, four ``bit-lets'' jet towards the malware, 
twirling into a tightly-closed circle, a signature of the Wrought-I/Orn hammer which you so 
cleverly installed before this encounter.

Numerous foes follow, each torn apart by the partnership of your ruthless trigger finger, 
and the weapon you endearingly refer to as ``Lucy''. Your conquest ends, rather heroically, 
to be greeted by the familiar, boisterous woman whose computer you have rid of the virtual 
invaders.

Left with the satisfaction of your coworker's appreciation, you feel your efforts echoing 
across the cubicle space. Friends of your new ally, Martha, now refer to you as a ``friend'', 
and promotions begin to feel inevitable. Some day, you too aspire to face the mysterious 
figure at the head of W.O.R.K., known to all as ``The C.E.O''.


\end{section}

\begin{section}{Story}

\hspace{1.5em}As a fresh-faced employee of the futuristic conglomerate known as ``W.O.R.K.'', 
your goal is to locate and destroy any viruses infecting the systems of your coworkers. You, 
along with any other of your cohorts, cleanse systems through the use of virtual reality. By 
cleaning each system, you gain cybernated components for your digital arsenal in addition to 
the networked affections of your coworkers.

Your own intentions, however, lie with working your way up the corporate ladder to eventually 
face the shadowy head of the company, ``The C.E.O.''. A figure masked in infamy, little is 
known of The C.E.O. aside from the effectual manner in which the company is run.

As you cut away infections from the workstations of your coworkers, the source and origin of 
the malodorous programming slowly reveals itself. The very same individual that determines the 
fate of your corporation is the one distributing the electronic diseases.
\newline

Determined to stop this proliferation, you commit yourself to reaching the corporate office 
and challenging your supervisor. System after system, your skill as a malware-slayer grows 
in addition to the supply of friends that begin to network and rally behind you.

Upon reaching the dark corporate ``castle'' of your dreaded executive handler, you enter to 
discover that the sinister force behind the computer plague is, itself, a computer. The final 
encounter of your journey, you quickly adorn your virtual equipment and venture into the 
mainframe of the baneful boss.

The swirling den of your former commander bids you death as you soon learn that there is no 
easy solution to defeating this monster. A quick snippet of code crafted by your affectionate 
coworkers materializes into a new shape in your hands, the ``Information Terminator''. Like a 
magnet to a monitor, you have but one option, you must lay the transmitter atop the C.E.O.'s 
sole weakness, the Malevolent Automation Node. As your mission becomes clear, lights flash across 
your screen, ``To defeat this otherwise impossible enemy, you must stick I.T. to the M.A.N.''.

The moment you strike the node with your transmitter, everything ceases to move. All lights and 
motion slowly fall out of existence. Like black paint flecked along a picture, the world around 
you gradually fades into nothingness. Soon, all that is left is the booming voice of the former 
Computerized Executive Officer repeating ``ERROR: SYSTEM MALFUNCTION: RECOMMEND REBOOT''. The 
cheers of your coworkers draw you back to reality, their exuberance can only mean one thing. As 
shapes and colors return to your vision, you glance at the monitor which had glared at you so 
threateningly before. A happy face, composed of few pixels illustrates the screen. Your victory 
is finally at hand.

\end{section}

\begin{section}{Mechanics}
\end{section}

\end{document}
