\documentclass[12pt]{report}
\usepackage{geometry}
\usepackage{titling}

\geometry{margin=1in}

\newcommand{\subtitle}[1]{%
	\posttitle{%
		\par\end{center}
		\begin{center}\large#1\end{center}
		\vskip0.5em}%
}

\title{Project Design and Direction}
\subtitle{CS 383 - Homework 1 - Group 3}
\author{Mason Fabel, Ronald Rodriguez, NAMES} % TODO: Everyone add your name to this list
\date{\today}

\begin{document}

\maketitle

\chapter{Individual Brainstorming}

% TODO: Each person needs a section here.

\begin{section}{Mason Fabel}
Our project is to be ``a turn-based co-operative multiplayer
'dungeon-crawl' role-playing game'', combining the ideas of an adventurous
science fiction setting and the premise of a lowly office worker rising
through the ranks to become the master of his own destiny.

While contemplating these goals, I came up with a few central concepts and
observations I would like to see incorporated into the final project.

\begin{subsection}{Tone and Theme}
While there are a lot of directions the overall tone and feel of this projecan take, I believe the best direction would be a lighthearted combination of comedy, in the vein of Douglass Adam's Hitchhiker's Guide, and
action/adventure, such as in Star Wars IV.

Taking this approach instead of a realistic approach gives the team more
freedom as to the shape of the final project. As we are not constrained by
accurately simulating reality, we can instead focus on creating a game that
is fun to play and avoid complicating areas that would not benifit from
such attentions.

Keeping a light tone will allow us to avoid dark or overly thoughtful
material. While this things have a time and a place, the purpose
of this project is to learn how to organize and create large pieces of
software, not to create an artistic of philosophical statement. By avoiding
such material, the focus is kept on the engineering, where it should be,
instead of the content, which is a matter for a different class.
\end{subsection}

\begin{subsection}{Content Generation}
The main purpose of this class is to learn software engineering, or how
to organize and build large software projects. While we have chosen to do
this in the form of a game, the purpose of this class is not game design,
but rather building a large application. Thus, whenever possible this
project should be moved away from creating game content and towards writing
code and organizing development.

To this end, I believe that as much game content should be generated on the
fly by the application as possible. This obviously covers simple things,
such as level layout, but it is not a hug jump of the immagination to take
this same aproach to many, if not all, of the elements that will appear in
the final product. Such examples may include NPCs, quest lines, or
equipment, but wherever we wish to have more than a handful of types this
sort of system is theoretically possible.

In addition to taking the burden of content generation away from the group
and giving it to the computer, this will have the further advantage of
allowing the size of the project to be scaled up (or down) by creating
different content generation methods for different sections of the final
game.

Finally, generating content will allow the final project, which is going to
be created by a small team in a short period of time, to have a lot of
content relative to programmer effort. Some finite amount of effort will
allow the final product to have a much larger amount of content, as
opposed to the more linear relationship involved when humans create
content by hand.
\end{subsection}

\begin{subsection}{NPC Social Networking}
While the previous sections discuss some general trends or directions I
like the final project to take after, this section contains my main idea
that I would very much like to see make it into the final project.

Traditionally, most RPGs have used a system based upon levels and
experience points. While this works well in a lot of settings, it seems to
me that this transfers poorly over to the concept of the main character
being an office worker.

Thus, I would like to make progression in our final project not dependant
upone collect experience or gaining levels, but rather on a ``social
network'' of NPCs, for lack of a better term.

The basic idea is pretty simple. You have some ``relationship value''
with each NPC character. This value can be some positive or negative value,
with positive values meaning they like the player and a negative value
meaning they do not.

In addition to relating to the player, NPCs would relate to each other by
some percentage value between $-100\%$ and $100\%$. This would then serve
to modify the player's relationship with that NPC.

For example, the player has a $+50$ relationship with NPC A and a $-15$
relationship with NPC B. NPC A has a $10\%$ relationship with NPC B, and
NPC B has a $50\%$ relationship with NPC A. Thus, the player's final
relationship value with NPC A is $50+0.1(-15)=48.5$, and the player's
final relationship value with NPC B is $-15+0.5(50)=10$.

As the number of NPCs increases, a increasingly complex and interesting web
of characters can be created.

Now suppose that NPCs were tied to plot events or player progression.
Specifically, in order to advance further in the game, the player needs
to be promoted within the office or organization they work for, and this
can only be accomplished by gathering favor with their coworkers and
managers.

Furthermore, NPCs could be made to ignore the player until they reached
some relationship threshhold, thus causing the player to build relationship
with that NPS's ``friends'' in order to interact with them.

Finally, some method of gaining relationship with NPCs would need to be
implemented, whether a traditional ``kill 30 grues'' quest or some other,
more original, method.

The result of this would be a varied and dynamic system (especially if this
network of NPCs was also generated by the game instead of being hard coded)
where the player is required to gain social goodwill until they finally
come to the attention of the top entities, thus recieving the final
promotion to CEO, or the final boss fight, or whatever the end game turns
out to be.
\end{subsection}

\end{section}

\chapter{Group Vision}

% TODO: Discuss Friday
% TODO: Appoint writers
% TODO: Actually have content here by Monday

\begin{section}{Setting}
\end{section}

\begin{section}{Story}
\end{section}

\begin{section}{Mechanics}
\end{section}

\end{document}
